\documentclass{article}

\usepackage[utf8]{inputenc}
\usepackage[english]{babel}
\usepackage{amsthm} %lets us use \begin{proof}
\usepackage{amssymb} %gives us the character \varnothing
\usepackage{xcolor}
\usepackage{float}
\usepackage{braket}
\usepackage{multirow}
\usepackage{array}
\usepackage{mathtools}

\title{Chem231B: Assignment \#3} % Title of the assignment

\begin{document}

\maketitle

\section*{Problems from Berry, Rice, and Ross}

a) Do problems 10, 12, 14, 15, and 18 from BRR. For reference data,
use the NIST website, Atomic Spectra Database. (This replaced Moore’s
tables, and the NBST became NIST). Also check out the Computational Chemistry
Comparison and Benchmark DataBase (cccbdb) from which I got some of the
basis-set results.
\\

10) Put entries in the following sets of configurations in order of
increasing energy. Then check your intuition with experimental results
at NIST Spectra Database.

\textcolor{red}{a) Li, $1s^22s, 1s^22p, 1s2s2p, 1s2s^2$}
\\

\textcolor{red}{b) C, $1s^22s^22p^2, 1s^22s2p^3, 1s^22s^22p3s, 1s2s^22p^3$}
\\

12) Explain why the second ionization potentail of lithium for the
process Li$^+$ \rightarrow $\text{Li}^{2+}$ + $e^-$, is less than that predicted for
a hydrogenlike atom with $Z=3$.

\textcolor{red}{The electronic repulsion from the two electrons within lithium
  ion leads to smaller second ionization potential than predicted by a hydrogenlike
  atom which only has one electron.}
\\

14) The outer electron of an alkali atom may be treated in an approximate way,
as if it were in a hydrogenic orbital. Suppose that one takes the quantum number
for the outer electron to be 2, 3, 4, 5, and 6, respectively, for Li, Na,
K, Rb, and Cs. What values must $Z$ be given to account for the observed
first ionization potentials of these atoms? Explain why they differ from unity.

\textcolor{red}{Look up first ionization potential (IP) energy on NIST and solve
  for $Z$ within the Rydberg equation where the HOMO may be taken as the IP energy
  from Koopman's Theorem.}
\\

\textcolor{red}{For lithium atom, the ionization potential energy is 5.3917eV}
\\

\textcolor{red}{
\begin{align*}
  \frac{13.6 Z^2}{2^2} & = 5.3917 \\
  Z^2 & = \frac{4*5.3917}{13.6} \\
  Z  & = \{\pm 1.259\}
\end{align*}
}

\textcolor{red}{$Z$ value for lithium is 1.259. Repeat for Na, K, Rb, and Cs}

\begin{table}[hbpt]
  \centering
  \begin{tabular}{cc}
    Atom & $Z$ \\
    \hline
    Li & 1.259 \\
    Na & 1.229\\
    K  & 1.130 \\
    Rb & 1.108 \\
    Cs & 1.070
  \end{tabular}
\end{table}

\textcolor{red}{When $Z=1$, this indicates an unscreened electron from the
  hydrogen atom. The $Z$ values are differ from unity because the nuclear attraction
  is increasing while at the same time, the screening effect does not exactly
  cancel with the attraction.}
\\

15) Use the Pauli exclusion principle and Hund's rules to find the number
of unpaired electrons and the term of lowest energy for the following atoms.

\textcolor{red}{a) P: [Ne]$3s^23p^3$; $L=0$ and $S=\frac{3}{2}$ and $J=L+S=\frac{3}{2}$;
  Term Symbol: $^4S_{\frac{3}{2}}$}
\\

\textcolor{red}{b) S: [Ne]$3s^23p^4$; $L=1$ and $S=1$ and $J=2$;
  Term Symbol: $^3P_2$}
\\

\textcolor{red}{c) Ca: [Ar]$4s^2$; $L=0$ and $S=0$ and $J=0$;
  Term Symbol: $^1S_0$}
\\

\textcolor{red}{d) Br: [Ar]$4s^23d^{10}4p^5$; $L=1$ and $S=\frac{1}{2}$
  and $J=\frac{3}{2}$; Term Symbol: $^2P_{\frac{3}{2}}$}
\\

\textcolor{red}{e) Fe: [Ar]$4s^23d^6$; $L=2$ and $S=2$ and $J=4$;
  Term Symbol: $^5D_4$}
\\

18) Derive the terms of the configuration $1s^22s^22p^63s^23p3d$,
of the silicon atom.

\textcolor{red}{$S=1$ and $L=3$ and $J=\{2,3,4\}$;
  Term Symbols: $^3F_2, ^3F_3, ^3F_4$}
\\

\noindent b) Deduce formulas for the atomic number of the noble gas atom
of the $n$-th row, and for the width of the $n$-th row.  Ignoring
relativity, in what row would the atom with 217 electrons be and
what is its ground-state configuration?  What would be its lowest
energy term?

\begin{table}[hbpt]
  \center
  \begin{tabular}{ccc}
    Row & Width & Atomic Number \\
    \hline
    1 & 2 & 2 \\
    2 & 8 & 10 \\
    3 & 8 & 18 \\
    4 & 18& 36 \\
    5 & 18& 54 \\
    6 & 32& 86 \\
    7 & 32& 118
  \end{tabular}
\end{table}

\textcolor{red}{Keeping with the existence of $f$-orbitals since
  $g$-orbitals have not been discovered yet. After the 6th row,
  add 32 electrons for the next noble gas atom. This will mean that the
  10-th row will be where the atom with 217 electrons is placed.}
\\

\noindent\textcolor{red}{$e^-$ configuration: [$^{214}$Noble Gas]$10s^29d^1$
  \\
  $S=1/2$ and $L=2$ and $J=3/2$; Term Symbol: $^2D_{\frac{3}{2}}$}
\\

\noindent c) Repeat (b) but with repulsion between electrons turned off, i.e.,
purely hydrogenic orbitals and energies.  Just give the formula for the
noble gas atoms, and state how many electrons are in the last shell
of element 217, and what is the letter for the largest ang mom that
might be occupied?

\textcolor{red}{If there's no electron interactions, lower quantum
  number orbitals will be filled first e.g. $3d$ orbitals are filled before
  $4s$ orbitals. Three electrons in the last shell of element 217.}
\\

\noindent\textcolor{red}{$e^-$ configuration: [$^{214}$Noble Gas]$9d^3$
  \\
  $S=3/2$ and $L=3$ and $J=3/2$; Term Symbol: $^4F_{\frac{3}{2}}$}

\end{document}
